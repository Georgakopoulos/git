\documentclass[aspectratio=169]{beamer}

% This document was written with the help of
% http://en.wikibooks.org/wiki/LaTeX/Presentations

% The theme of this presentation
\usetheme{default}

\definecolor{blue}{HTML}{196284}

\setbeamercolor{title}{fg=blue}
\setbeamercolor{titlelike}{fg=blue}

% Makes more space between new lines
\setlength{\parskip}{10pt plus 1pt minus 1pt}

% Open Sans Package
\usepackage[default,scale=0.95]{opensans}
\usepackage[T1]{fontenc}



% Title page
\title[Learn You some GIT] % (optional, only for long titles)
{Learn You some GIT}
\author[Florian Willich]{Florian Willich}
\institute[BIT]
{
  Quality and Usability Lab\\
  Berlin Institute of Technology
}
\date{\today}

\begin{document}

\frame{\titlepage}

\begin{frame}
\frametitle{Table of Contents}
\tableofcontents[currentsection]
\end{frame}

\section{Create Repositories}
\begin{frame}

\frametitle{Create Repositories}

\$ \textcolor{blue}{git init} \textit{project-name}\\
Creates a new local repository with the specified name

\$ \textcolor{blue}{git clone} \textit{url}\\
Downloads a project and its entire version history but only the master branch!

\$ \textcolor{blue}{git fetch} \textit{remote-branch}/\textit{local-branch}\\
lets you fetch the remote branch and create a local branch

\$ \textcolor{blue}{git checkout} \textit{local-branch}\\
switch to the branch\\

\end{frame}

\section{Make Changes}
\begin{frame}
  
\frametitle{Make Changes}

\$ \textcolor{blue}{git status}\\
Most important command! Lists all new or modified files to be committed

\$ \textcolor{blue}{git add} \textit{file}\\
Snapshots the file in preparation for versioning

\$ \textcolor{blue}{git commit -m} "\textit{descriptive message}"\\
Records file snapshots permanently in version history

\$ \textcolor{blue}{git commit -am} "\textit{descriptive message}"\\
Snapshots all tracked files in preparation for versioning \& records file snapshots permanently in version history

\end{frame}

\section{Group Changes}
\begin{frame}

\frametitle{Group Changes}

\$ \textcolor{blue}{git branch}\\
Lists all local branches in the current repository

\$ \textcolor{blue}{git branch} \textit{branch-name}\\
Creates a new branch with the specified branch name

\$ \textcolor{blue}{git checkout} \textit{branch-name}\\
Switches to the specified branch and updates the working directory

\$ \textcolor{blue}{git merge} \textit{branch-name}\\
Combines the specified branch's history into the current branch

\$ \textcolor{blue}{git branch -d} \textit{branch-name}\\
Deletes the specified branch

\end{frame}

\section{Suppress Tracking}
\begin{frame}

\frametitle{Suppress Tracking}

By creating a file called \textcolor{blue}{.gitignore} (yes its a hidden file) in the root directory, you can specify all the files you want git to ignore.

Examples for files you don't want to track:

\begin{itemize}

\item *.log
\item *.config
\item my-secret-passwords.secret
\item Any IDE related files

\end{itemize}

\end{frame}

\section{Save Fragments}
\begin{frame}

\frametitle{Save Fragments}

\$ \textcolor{blue}{git stash}\\
Temporarily stores all modified tracked files

\$ \textcolor{blue}{git stash pop}\\
Restores the most recently stashed files

\end{frame}

\section{Synchronize Changes}
\begin{frame}

\frametitle{Synchronize Changes}

\$ \textcolor{blue}{git pull}\\
Downloads bookmark history and incorporates changes

\$ \textcolor{blue}{git push}\\
Uploads all local branch commits

\$ \textcolor{blue}{git merge} \textit{branch}\\
Merges the specified branch changes into the the branch you're currently in

\end{frame}

\section{The five daily commands}
\begin{frame}

\frametitle{The five daily commands}

\begin{itemize}
\item \$ \textcolor{blue}{git status}\\
\item \$ \textcolor{blue}{git pull}\\
\item \$ \textcolor{blue}{git add} \textit{file}\\
\item \$ \textcolor{blue}{git commit -m} "\textit{descriptive message}"\\
\item \$ \textcolor{blue}{git push}\\
\end{itemize}

And please don't use \textcolor{blue}{git commit -am} "\textit{message}"!

\end{frame}

\section{miscellaneous}
\begin{frame}

\frametitle{miscellaneous}

\item \$ \textcolor{blue}{git checkout} \textit{hash}\\
Use this only to look up the current state of the commit.

git checkout -b old-state 0d1d7fc32

\end{frame}

% etc
\end{document}